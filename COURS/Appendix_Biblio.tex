

\chapter{Bibiographie}

\subsection*{Ouvrages en fran�ais}

\begin{description}

\item{\bf Thermodynamique, 1�re et 2�me ann�e (collection H-Pr�pa)}

Bonne r�f�rence, notations et d�finitions proches de celles de ce cours (mais plan diff�rent). 



\item{\bf Thermodynamique : une approche pragmatique (Cengel, Boles \& Lacroix, Ed. de Boeck)}

Approche pragmatique "� l'am�ricaine", beaucoup d'exemples et d'exercices. Attention : notations et conventions diff�rences par rapports aux ouvrages fran�ais (notamment signe du travail).


\item{\bf Thermodynamique : fondements et applications (Perez, Ed. Dunod) }

Approche classique "� la fran�aise", assez complet.


%Approche physique

\item{\bf Thermodynamique (Diu, Guthmann, Lederer \& Roulet)} 

Approche formalis�e, correspond au programme de L3 m�ca / M1 m�ca. 
 

%\subsubsection*{Thermodynamique : Bases et applications (Foussard, Julien, Math�)  }

%\subsubsection*{ La thermodynamique : des principes aux applications (Cheze, Bauer)}

\end{description}

\subsection*{Autres lectures conseill�es :}


\begin{description}
\item{\bf 
Cours de physique de Feynman : Tome 1 ; M�canique (R. Feynman, Dunod )
}

Pour votre culture g�n�rale : l'approche tr�s originale de Richard Feynman, prix Nobel de physique.

\item{\bf  L'entropie et tout �a : le roman de la thermodynamique.
(P. Depondt ;  Vuibert)}

Un "roman" facile a lire et int�ressant.
\end{description}

\subsection*{Sur internet}

\subsubsection{Cours en ligne :}
{\small

\verb| http://manuel.marcoux.pagesperso-orange.fr/ |

\verb| http://www.fast.u-psud.fr/~doumenc/la200/CoursThermodynamique_L2.pdf |

\verb| http://www.fast.u-psud.fr/~doumenc/la200/CoursThermique_L2.pdf |

\verb| http://www.sciences.univ-nantes.fr/sites/claude_saintblanquet/thermo2005/th2004dex.htm |

\verb| http://www-ipst.u-strasbg.fr/cours/thermodynamique/ |

}


{\bf Wikip�dia :} 

Pages "Thermodynamique", "Energie", "Chaleur", "Temp�rature", "Transferts thermiques", "Entropie", ... 


{ \bf Le forum de r�f�rence des �tudiants en physique :}

{\small
\verb| forums.futura-sciences.com/physique |
}

